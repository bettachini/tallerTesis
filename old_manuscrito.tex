\documentclass[a4paper,twoside,openright]{book}


% \usepackage{fullpage}
\usepackage[margin=1.1in]{geometry}


%% Spanska!
\usepackage[utf8]{inputenc}
\usepackage[T1]{fontenc}    % 8-bit encoding. Acentuados como único caracter.
\usepackage[spanish, es-tabla]{babel}
\def\spanishoptions{argentina}


%% inclusión de gráficas
\usepackage[pdftex]{graphicx}           % instalar ghostscript-x para que el dvi muestre los eps
% \usepackage{graphicx}           % instalar ghostscript-x para que el dvi muestre los eps
\graphicspath{ {./graphs/} {../}}
\usepackage{epstopdf}


%% tablas: multirow
\usepackage{multirow}
\usepackage{booktabs}


%% Matemática
\usepackage{amsmath}


%% Subfloats
\usepackage{subcaption}
% \usepackage{subfig} % deprecated https://en.wikibooks.org/wiki/LaTeX/Floats,_Figures_and_Captions


% Interlineado 1,5
% \renewcommand{\baselinestretch}{1.5}
\renewcommand{\baselinestretch}{1.7}


%% encadezado páginas
\usepackage{fancyhdr}
\pagestyle{fancy}
% with this we ensure that the chapter and section
% headings are in lowercase.
\renewcommand{\chaptermark}[1]{\markboth{#1}{}}
\renewcommand{\sectionmark}[1]{\markright{\thesection\ #1}}
\fancyhf{} % delete current setting for header and footer
\fancyhead[LE,RO]{\bfseries\thepage}
\fancyhead[LO]{\bfseries\rightmark}
\fancyhead[RE]{\bfseries\leftmark}
\renewcommand{\headrulewidth}{0.5pt}
\renewcommand{\footrulewidth}{0pt}
%\addtolength{\headheight}{3.58337pt} % make space for the rule
\addtolength{\headheight}{15.2pt} % make space for the rule
\fancypagestyle{plain}{%
\fancyhead{} % get rid of headers on plain pages
\renewcommand{\headrulewidth}{0pt} % and the line
}


%% environment abstract para documentclass book
\newenvironment{abstract}%
{\cleardoublepage\null \vfill\begin{center}%
\thispagestyle{empty}
\bfseries \abstractname \end{center}}%
{\vfill\null}


%% Biblatex
%\usepackage[hyperref=true,
%	url=false,
%	isbn=false,
%	backref=true,
%	% style=custom-numeric-comp,
%	style=numeric,
%	citereset=chapter,
%	maxcitenames=7,
%	maxbibnames=100,
%	block=none,
%	sorting=none,
%	backend=biber]{biblatex}
%\DefineBibliographyStrings{spanish}{}
%\usepackage{textgreek}	% http://tex.stackexchange.com/questions/107352/non-ascii-characters-in-biblatex
%\usepackage{csquotes}
%% no urldate
%\DeclareSourcemap{
%  \maps[datatype=bibtex]{
%    \map[overwrite=true]{
%      \step[fieldset=urldate, null]
%    }
%  }
%}
%\addbibresource{thesis.bib}


%% biblatex
\usepackage[style=numeric, backend=biber, sorting= none, url= false, maxnames=20]{biblatex}
\DefineBibliographyStrings{spanish}{}
\usepackage{csquotes}
\usepackage{textgreek} % https://tex.stackexchange.com/questions/107352/non-ascii-characters-in-biblatex 
\addbibresource{bucher thesis.bib}
\addbibresource{NLFO-PCFs.bib}
\addbibresource{Secure Optical Network.bib}


%% Nomenclatura
\usepackage[spanish, intoc]{nomencl}
\makenomenclature


%% URLs
\usepackage{color}
\definecolor{DarkBlue}{rgb}{0,0,0.4}
\usepackage[hidelinks]{hyperref}
% \usepackage[colorlinks=true,urlcolor=DarkBlue]{hyperref}
% \usepackage{url}


%% Unidades
\usepackage[locale=FR, per-mode=fraction, separate-uncertainty=true]{siunitx}
\newcommand{\unit}[1]{\ensuremath{\, \mathrm{#1}}}	% ¿Que hace esto?
\sisetup{detect-all}	% ¿Que hace esto?


%% para doublebox linkSim
\usepackage{fancybox}


%% reinicio del contador de capítulos ante \part
%% https://tex.stackexchange.com/questions/35782/how-to-split-a-latex-document-using-parts-and-chapters
%% Comment the following to have chapters numbered without interruption (numbering through parts)
\makeatletter\@addtoreset{chapter}{part}\makeatother%


\setcounter{tocdepth}{2}	%% tabla de contenidos hasta nivel 3
% \setcounter{tocdepth}{1}	%% tabla de contenidos hasta nivel 2


%% \textsubscript \textsuperscript
\usepackage{changes}
% \usepackage{fixltx2e} 


%% Glosario
\usepackage[acronym,toc,nonumberlist,translate=babel]{glossaries}
\setacronymstyle{short-long-desc}
\include{glosario}
\makeglossaries



\begin{document}

%% paginas titulo según
%% http://www.df.uba.ar/academica/carrera-de-doctorado/pautas
\begin{titlepage}
	\begin{center}
		\includegraphics[width=0.25\textwidth]{Logo}\\[1cm]
		\textbf{\LARGE Universidad de Buenos Aires}\\
		Facultad de Ciencias Exactas y Naturales\\
		%Departamento de Física\\[1cm]
  \textit{
		\textbf{Monitoreo cruzado de escalas de tiempo atómico nacionales\\
		Extensión del rango temporal de}}\\
		% \textbf{Simulación de capa física de red óptica pasiva encriptada}\\[1cm]
		Trabajo final de la Especilización\\
		% Doctor de la Universidad de Buenos Aires en el área Ciencias Físicas\\[1cm]
		\textit{\textbf{Víctor A. Bettachini}}\\[1cm]
		% por \textit{\textbf{Víctor A. Bettachini}}\\[1cm]
	\end{center}
		Director de Tesis: \\
		%Director de Tesis: Dr. Diego Fernando Grosz\\
		%Director Asistente: Dr. José Ignacio Álvarez-Hamelin\\
		%Consejera de Estudios: Dra. Victoria Bekeris\\[1cm]
		%Lugar de Trabajo: Laboratorio de Optoelectrónica\\
		%\hspace{3.9cm} Instituto Tecnológico de Buenos Aires\\[1cm]
		\textit{\textbf{xx de xx de 2024}}
\end{titlepage}

\newpage
\thispagestyle{empty}
% \begin{abstract}
% 	Blah.
% \end{abstract}

%% A continuación de la Carátula, debe figurar un resumen del trabajo, junto con palabras claves asociadas.
% \thispagestyle{empty}
\begin{abstract}
	Blah.
\end{abstract}
Palabras clave?

\newpage
\thispagestyle{empty}
%% Luego, en página aparte, el título de la Tesis, el resumen y las palabras claves, traducido al inglés.	
{\cleardoublepage\null \vfill\begin{center}%
\thispagestyle{empty}
\bfseries Abstract \end{center}}%
{\vfill\null}
Keywords?

\newpage
\thispagestyle{empty}

\tableofcontents

% Print your own papers.
\chapter{Publicaciones y presentaciones del autor}
Las siguientes son publicaciones y presentaciones en conferencias en las que dan sustente a sendas partes que componen esta tesis.

\section*{Parte I}
\cite{caldarola_fuente_2009}
\cite{caldarola_fuente_2010-1}
\cite{rieznik_pcf-based_2010}
\cite{rieznik_optimum_2012}
\cite{caldarola_high-speed_2012}


\section*{Parte II} 
\cite{ortega_point--point_2011}
\cite{ortega_altas_2010}
\cite{ortega_cdma_2012}
\cite{ortega_hamming-weight_2012}
\cite{alvarez-hamelin_device_2013} % Patente
\cite{ortega_encrypted_2014}


\section*{Durante el doctorado no relacionadas} 
\cite{pasquini_memory_2007}
\cite{beget_sailhflood:_2013}



%\begin{refsection}
%% If you print a bibliography within this section, only citations within this refsection will be printed.
%
%% Option 1: Make nocite for all of your papers.
%% Options 2 would be a seperate file which contains all of your papers.
%%\section*{Parte I}
%\cite{caldarola_fuente_2009}
%\cite{caldarola_fuente_2010-1}
%\cite{rieznik_pcf-based_2010}
%\cite{rieznik_optimum_2012}
%\cite{caldarola_high-speed_2012}
%
%% \section*{Parte II} 
%\cite{ortega_point--point_2011}
%\cite{ortega_altas_2010}
%\cite{ortega_cdma_2012}
%\cite{ortega_hamming-weight_2012}
%\cite{alvarez-hamelin_device_2013} % Patente
%\cite{ortega_encrypted_2014}
%
%% \section*{Durante el doctorado no relacionadas} 
%\cite{pasquini_memory_2007}
%\cite{beget_sailhflood:_2013}
%
%\defbibnote{myPrenote}{
%	Esta tesis se basa en las siguientes publicaciones.
%%	y presentaciones y publicaciones del autor separadas por su relación a la temática de la primer o segunda parte.
%}
%%\defbibnote{myPostnote}{
%%	A bunch of papers are still in print and not yet published.
%%}
%\printbibliography[
% heading=bibintoc,
%	title={Contribuciones del autor},
% prenote=myPrenote,
% %postnote=myPostnote
%]
%\end{refsection}
%




\printglossaries

%% Sección de nomenclatura
%% makeindex phdthesis.nlo -s nomencl.ist -o phdthesis.nls
% \printnomenclature
% \chapter*{Simbología utilizada\markboth{Simbología utilizada}{Simbología utilizada}}
% \addcontentsline{toc}{chapter}{Simbología utilizada}
% \begin{description}
% 	\item[$a_{\Delta}$:] \emph{Parámetro de la red triangular de vórtices}. Longitud de los vectores de la celda unidad de una red bidimensional de \emph{Bravais} triangular (o hexagonal).
% \end{description}


\newpage
\thispagestyle{empty}

\chapter{Preámbulo: dos lineas de trabajo... ¿y su unión?}
Presentar las dos lineas de trabajo.
Algún argumento de posible unión.
Explicar que finalmente no se se unieron y se presentan en la tesis por separado.


\mainmatter

\part{Fuente de pulsos ultra cortos sintonizable en longitud de onda}

% the scientific background of your work
% Introducción
\include{sci_Background}

% the goals posed
% dentro de Introducción

% the experimental and theoretical methods applied
% Modelo teórico
\include{theory}

% Simulación
\include{simulacion}

% Experimento
\input{setup}	% ESTA ES LA FORMA CORRECTA, INCLUYENDO \chapter dentro del archivo

% the results and
\chapter{Resultados}

% \chapter{Demostración experimental de sintonización rápida en el infrarojo (2012PapersInPhyics)}
\input{2012_PapersInPhysics}

% \chapter{Simulación de filtrado de pulsos visibles (2010_NP10)}
\input{2010_NP10}

% \chapter{Simulación de control de pulsos visible e infrarojos}
\input{twoPulses}

% their evaluation as well as
% an appropriate summary

% \chapter{¿Qué se logró y qué queda pendiente?}
\input{balance_Pendientes}





\part{Red óptica de acceso con seguridad en capa física}


\chapter{CDMA y concepto sustentando la idea}

% \chapter{Código de simulación del \emph{link}}
% Volcar escrito sobre código \emph{noisesim}.

\input{redOptica}

\input{linkSim}

\chapter{Resultados de simulación}

\chapter{Desarrollos posteriores - Implementación física}
\section{Implementación en FPGA}
\section{Implementación en sonido}


% \include{test}

\part{Apendices}
\appendix 


%% Print your own papers.
%\begin{refsection}
%% If you print a bibliography within this section, only citations within this refsection will be printed.
%
%% Option 1: Make nocite for all of your papers.
%% Options 2 would be a seperate file which contains all of your papers.
%\nocite{caldarola_fuente_2009}
%\cite{caldarola_fuente_2010}
%
%
%\defbibnote{myPrenote}{
%    Some words before I show you the list of my own papers.
%}
%\defbibnote{myPostnote}{
%    A bunch of papers are still in print and not yet published.
%}
%\printbibliography[
%    heading=bibintoc,
%    title={Publicaciones y presentaciones relacionadas a la tesis bib},
%    prenote=myPrenote,
%    postnote=myPostnote
%]
%\end{refsection}

%% HAY QUE HACERLO

%% bibliografía
\printbibliography[heading=bibintoc]

\printglossaries

\backmatter
% \chapter{Glosario}



\chapter{Agradecimientos}
% \chapter{Agradecimientos\markboth{Agradecimientos}{Agradecimientos}}

\end{document}
